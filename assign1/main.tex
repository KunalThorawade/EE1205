\let\negmedspace\undefined
\let\negthickspace\undefined
\documentclass[journal,12pt,twocolumn]{IEEEtran}
\usepackage{cite}
\usepackage{amsmath,amssymb,amsfonts,amsthm}
\usepackage{algorithmic}
\usepackage{graphicx}
\usepackage{textcomp}
\usepackage{xcolor}
\usepackage{txfonts}
\usepackage{listings}
\usepackage{enumitem}
\usepackage{mathtools}
\usepackage{gensymb}
\usepackage[breaklinks=true]{hyperref}
\usepackage{tkz-euclide} % loads  TikZ and tkz-base
\usepackage{listings}
\usepackage{gvv}


\newtheorem{theorem}{Theorem}[section]
\newtheorem{problem}{Problem}
\newtheorem{proposition}{Proposition}[section]
\newtheorem{lemma}{Lemma}[section]
\newtheorem{corollary}[theorem]{Corollary}
\newtheorem{example}{Example}[section]
\newtheorem{definition}[problem]{Definition}

\newcommand{\BEQA}{\begin{eqnarray}}
\newcommand{\EEQA}{\end{eqnarray}}
\newcommand{\define}{\stackrel{\triangle}{=}}
\theoremstyle{remark}
\newtheorem{rem}{Remark}

\graphicspath{./figs/}

%\bibliographystyle{ieeetr}
\begin{document}
%

\bibliographystyle{IEEEtran}


\vspace{3cm}

\title{
%	\logo{
Assignment-1 

\large{EE:1205 Signals and Systems}

Indian Institute of Technology, Hyderabad
%	}
}
\author{Kunal Thorawade

EE23BTECH11035
}	

\maketitle


\newpage

%\tableofcontents

\bigskip
 
\renewcommand{\thefigure}{\theenumi}
\renewcommand{\thetable}{\theenumi}
%\renewcommand{\theequation}{\theenumi}
\begin{flushleft}

\section{\Large Question:}  Ramkali saved Rs 5 in the first week of a year and then increased her weekly savings by Rs 1.75. If in the $n$th week, her weekly savings become Rs 20.75, find $n$.

\section{\Large Solution:} 
\begin{table}[ht]
    \centering
    \begin{tabular}{|c|c|}
        \hline
        Parameter & Value \\
        \hline
        First term of AP (x(0)) & 5 \\
        \hline
        Common difference (d) & 1.75 \\
        \hline
        $n^{th}$ term of AP (x(n)) & 20 \\
        \hline
    \end{tabular}
    \vspace{2mm}
    \caption{Parameter List}
    \label{tab:simple}
\end{table}


\end{flushleft}
\begin{center}

\begin{align} 
x(n) &= x(0) + (n)(d)
\\ 20.75 &= 5 + (n)(1.75)  
 \\ \implies 20.75 - 5 &= (n)( 1.75)
\\ \implies 15.75 &= (n)(1.75)
\\ \implies n &= \frac{15.75}{1.75} = \frac{1575}{175} 
\\ \implies n &= \frac{63}{7} = 9
\\ \implies n &= 9
\end{align}
\end{center}


\begin{center}
\textbf{Hence, $n$ is 9.}
\vspace{1mm}
\begin{align}
x(n) &= x(0) + (n)(d)
\\ \implies x(n) &= 5 + 1.75(n)
\end{align}
\\The Z-transform of a sequence $x(n)$ is given by:
\begin{align}
  X(z) &= \sum_{n=1}^{\infty} (5 + 1.75n)z^{-n}  
  \\ X(z) &= \sum_{n=1}^{\infty} 5z^{-n} + \sum_{n=1}^{\infty} 1.75nz^{-n}
  \\ X(z) &= 5U(z) + 1.75(z)\frac{d}{dz} U(z)
  \\ X(z) &= \frac{5z^{-1}}{1-z^{-1}}+\frac{1.75z^{-1}}{(1-z^{-1})^{2}}
\end{align}
\begin{figure}
    \centering
    \includegraphics[width = 8cm]{figs/fig1.jpg}
    \caption{Plot of $x(n) = 5 + 1.75n$}
    \label{fig:enter-label}
\end{figure}

 \begin{align}
        f(n) = 
        \begin{cases}
            5 + 1.75n, & \text{if } n \geq 0 \\
            0, & \text{if } n < 0
        \end{cases}
    \end{align}

\vspace{4mm}
Given that \( n > 0 \),
    \begin{align}
    X(z) &= \sum_{n=1}^{\infty} (5 + 1.75n) \cdot z^{-n}
    \end{align}

\[ROC : |z| > 1\]

\end{center}
\end{document}
